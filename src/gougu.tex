% 请使用xelatex进行编译
% gougu.tex
% 勾股定理
\documentclass[UTF8]{ctexart}

\title{杂谈勾股定理}
\author{张三}
\date{\today}

\bibliographystyle{plain}

\begin{document}
\maketitle
\tableofcontents
\section{勾股定理在古代}
西方称勾股定理为毕达哥拉斯定理,将勾股定理的发现归功为公元前 6 世纪的毕达哥拉斯学派。毕达哥拉斯学派没有书面著作,该定理的严格表述和证明则见于欧几里得
\footnote{欧几里得,约公元前330--275年}
《几何原本》的命题
 47:“直角三角形斜边上的正方形等于两直角边上的两个正方形之和。”证明是用面积做的。
\section{勾股定理的近代形式}
\bibliography{math}

\end{document}